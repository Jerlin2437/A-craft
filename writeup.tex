\documentclass{article}
\usepackage[margin=1in]{geometry} % Set 1-inch margins on all sides

\title{Exploring A star and Star Craft}
\author{Jerlin Yuen and Mick Estiler}
\date{\today} % or specify a date

\begin{document}

\maketitle

\section{Introduction}
To preface, we have decided to use DFS as our main method to generate our maze. We have used the repository, https://github.com/algoprog/Laby, to base our own implementation of the maze generation in our project. We mainly used Chris's implementation since it also included a start and end node within the 2d 101x101 maze matrix.

\section{Question 1}
Here, we will explore a bit about the A star algorithm.

\paragraph{Part A}

We want to explore the reasoning of why A* moves east instead of north in Figure 8 of the Homework 1 writeup.

First, we want to recognize that the algorithm does not know which nodes are blocked and which are unblocked. Only neighbouring nodes are visible (immediate north, south, west, east nodes). In figure 8, there are 3 possible nodes that can be traversed at the start, each of which is unblocked. They are: \((E, 1)\), \((E, 3)\), \((D, 2)\) The heuristic \(h(n)\) that will be used will be Manhattan distances. \(g(n)\) will simply be the total traveled distance. 

At the start, \(g(n)\) is obviously 0. \(f(n) = h(n) + g(n)\), but \(g(n) = 0\), so \(f(n) = h(n)\). A* explores the path with the lowest \(h(n)\). At \((E,1)\), \(h(n) = 4\). At \((D,2)\), \(h(n) = 4\). At \((E,3)\), \(h(n) = 2\). Obviously, \(h(n) = f(n)\) at \((E,3)\) is the smallest value so the agent moves to the east first.

\paragraph{Part B}
A subsection within Section 2.

\section{Conclusion}
Conclude your document here.

\end{document}
